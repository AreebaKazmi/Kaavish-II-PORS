\section{Conclusion}
This thesis addressed the problem of product curation in fashion retail, i.e. selecting, organizing, and presenting the outfits in their best possible combinations, to the consumers. To solve this problem we presented a web-application that uses an AI model to recommend outfits to the shopper based on their input image. 

The web application was developed using TypeScript and React at the client-side, while at the server-side, Python and Django were used.  Whereas, React Apollo and Graphene Django were used for interacting with and creating the GraphQL API, respectively. 

The web application uses a  deep learning model to retrieve clothing items that are visually similar to those in the user uploaded image. The model uses a Convolutional Neural Network (CNN) to extract a feature vector of each item's image that is available in our store, which currently, consists of some of the local stores like J., Furor, Zellbury, Export Leftovers, and a dummy store. Similarly, a feature vector is extracted from the input image by the user and the items with the most similar feature vectors are displayed on our web-application. Our feature network consists of a ResNet50 model trained on the DeepFashion dataset - the largest and best-annotated dataset in the domain of fashion. Each image is then converted into a 512-dim feature vector, which captures the unique visual information of the clothing item in the image, including its style, color, and pattern. Clothing items whose feature vectors have minimum Euclidean distance from the query feature vector are then displayed to the user.
\section{Future Work}

In the future, first of all, we would look into the possibility to configure the recommendation model as a micro-service API and deploy the web application to a Web Server to make it feasible and accessible to both, the shoppers and the retailers.

Personalized Outfit Recommendation web application has a great potential to incorporate several enhancements in the future. Most importantly, refining the results of our AI model. Once the application is deployed and available publicly, there is a chance of collaborating with more local brands, meaning more data to feed into the AI model, consequently improving the recommendations. Additionally, the model will be trained in eastern fashion to make eastern recommendations more intelligent and robust, which is one of the novelties of this thesis. Furthermore, if the likes, dislikes, and purchase history of the customer are taken into account, it will further help to make the experience and recommendations more personalized for each customer. In addition to this, it can be further extended to the complete range of men's and women's clothing.

Lastly, a portable version of web application or a mobile application for Android and iOS would do a great deal in reaching out to a larger audience.